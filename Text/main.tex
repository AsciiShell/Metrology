%!TEX TS-program = xelatex

% Шаблон документа LaTeX создан в 2018 году
% Алексеем Подчезерцевым
% В качестве исходных использованы шаблоны
% 	Данилом Фёдоровых (danil@fedorovykh.ru) 
%		https://www.writelatex.com/coursera/latex/5.2.2
%	LaTeX-шаблон для русской кандидатской диссертации и её автореферата.
%		https://github.com/AndreyAkinshin/Russian-Phd-LaTeX-Dissertation-Template

\documentclass[a4paper,14pt]{article}

\input{data/preambular.tex}
\begin{document} % конец преамбулы, начало документа
\begin{titlepage}
	\begin{center}
		ФЕДЕРАЛЬНОЕ  ГОСУДАРСТВЕННОЕ АВТОНОМНОЕ \\
		ОБРАЗОВАТЕЛЬНОЕ УЧРЕЖДЕНИЕ ВЫСШЕГО ОБРАЗОВАНИЯ\\
		«НАЦИОНАЛЬНЫЙ ИССЛЕДОВАТЕЛЬСКИЙ УНИВЕРСИТЕТ\\
		«ВЫСШАЯ ШКОЛА ЭКОНОМИКИ»
	\end{center}
	
	\begin{center}
		\textbf{Московский институт электроники и математики}
		
		\textbf{Им. А.Н.Тихонова НИУ ВШЭ}
		
		\textbf{Департамент электронной инженерии}
	\end{center}
	\vspace{1ex}	
	\begin{center}
		\textbf{Домашнее задание на тему}
	\end{center}	
	\vspace{1ex}
	\begin{center}
		\textbf{{\Large <<Выявление медленно меняющихся систематических погрешностей с помощью комбинаторного критерия>>}}
	\end{center}	
	\vspace{2ex}
	\begin{center}
\textbf{	
		по дисциплине 
		<<Электротехника, электроника и метрология», раздел <<Метрология>>}


	\end{center}
	\vspace{2ex}
	\begin{flushright}
		Выполнили
		
		Рабочая группа №6:
		
		Подчезерцев Алексей Евгеньевич \\ (БИВ172)
		
		Солодянкин Андрей Александрович \\ (БИВ172)
		
		Соловьев Александр Валерьевич \\ (БИВ172)
	\end{flushright}
	\vspace{3ex}
	\vfill
	\begin{center}
		Москва \the\year \, г.
	\end{center}
\end{titlepage}

\section{Аннотация}



\section{Методика обработки}

Методика обработки выборки на наличие выбросов по критерию Шарлье.
\begin{itemize}
	\item Определение критерия Шарлье. Если сомнительным в ряду наблюдений является один результат, то
	
	$$n[1 - \text{Ф}(K_{\text{Ш}})] = 1$$
	
	Отсюда 
	
	$$K_{\text{Ш}} =  \text{Ф'}(\frac{n-1}{n})$$
	
	где  $\text{Ф}$ -- функция ошибки, $\text{Ф'}$ -- функция, обратная функции ошибки.
	
	\item Определение среднего значения результатов измерения:
	
	$$\overline{X} = \dfrac{1}{n}\sum_{i=1}^{n}x_i$$
	
	\item Определение среднего квадратичного отклонения $s(x)$:
	
	$$\overline{X} = \sqrt{\dfrac{1}{n-1}\sum_{i=1}^{n}(x_i - (\overline X))^2}$$
	
	\item Сравниваются значения
	
	$$|x_{\text{сомнит}} - (\overline X)| \text{ ? } s(x)K_{\text{Ш}}$$
	
	Если $|x_{\text{сомнит}} - (\overline X)|$ больше, то результат отбрасывается, иначе остается.
	
	\item Данные операции следует повторить на полученной выборке до тех пор, пока все значения все не будут удовлетворять критерию Шарлье или в выборке не останется элементов.
\end{itemize}

\section{Описание програмного компонента}

\section{Результаты обработки данных}

\section{Библиографический список}

\section{Заключение}

\section{Приложения}
\end{document} % конец документа

