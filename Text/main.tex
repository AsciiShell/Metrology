%!TEX TS-program = xelatex

% Шаблон документа LaTeX создан в 2018 году
% Алексеем Подчезерцевым
% В качестве исходных использованы шаблоны
% 	Данилом Фёдоровых (danil@fedorovykh.ru) 
%		https://www.writelatex.com/coursera/latex/5.2.2
%	LaTeX-шаблон для русской кандидатской диссертации и её автореферата.
%		https://github.com/AndreyAkinshin/Russian-Phd-LaTeX-Dissertation-Template

\documentclass[a4paper,14pt]{article}

\input{data/preambular.tex}
\begin{document} % конец преамбулы, начало документа
\begin{titlepage}
	\begin{center}
		ФЕДЕРАЛЬНОЕ  ГОСУДАРСТВЕННОЕ АВТОНОМНОЕ \\
		ОБРАЗОВАТЕЛЬНОЕ УЧРЕЖДЕНИЕ ВЫСШЕГО ОБРАЗОВАНИЯ\\
		«НАЦИОНАЛЬНЫЙ ИССЛЕДОВАТЕЛЬСКИЙ УНИВЕРСИТЕТ\\
		«ВЫСШАЯ ШКОЛА ЭКОНОМИКИ»
	\end{center}
	
	\begin{center}
		\textbf{Московский институт электроники и математики}
		
		\textbf{Им. А.Н.Тихонова НИУ ВШЭ}
		
		\textbf{Департамент электронной инженерии}
	\end{center}
	\vspace{1ex}	
	\begin{center}
		\textbf{Домашнее задание на тему}
	\end{center}	
	\vspace{1ex}
	\begin{center}
		\textbf{{\Large <<Выявление медленно меняющихся систематических погрешностей с помощью комбинаторного критерия>>}}
	\end{center}	
	\vspace{2ex}
	\begin{center}
\textbf{	
		по дисциплине 
		<<Электротехника, электроника и метрология», раздел <<Метрология>>}


	\end{center}
	\vspace{2ex}
	\begin{flushright}
		Выполнили
		
		Рабочая группа №6:
		
		Подчезерцев Алексей Евгеньевич \\ (БИВ172)
		
		Солодянкин Андрей Александрович \\ (БИВ172)
		
		Соловьев Александр Валерьевич \\ (БИВ172)
	\end{flushright}
	\vspace{3ex}
	\vfill
	\begin{center}
		Москва \the\year \, г.
	\end{center}
\end{titlepage}

\section{Аннотация}



\section{Методика обработки}

Методика обработки выборки на наличие выбросов по критерию Шарлье.
\begin{itemize}
	\item Определение критерия Шарлье. Если сомнительным в ряду наблюдений является один результат, то
	
	$$n[1 - \text{Ф}(K_{\text{Ш}})] = 1$$
	
	Отсюда 
	
	$$K_{\text{Ш}} =  \text{Ф'}(\frac{n-1}{n})$$
	
	где  $\text{Ф}$ -- функция ошибки, $\text{Ф'}$ -- функция, обратная функции ошибки.
	
	\item Определение среднего значения результатов измерения:
	
	$$\overline{X} = \dfrac{1}{n}\sum_{i=1}^{n}x_i$$
	
	\item Определение среднего квадратичного отклонения $s(x)$:
	
	$$\overline{X} = \sqrt{\dfrac{1}{n-1}\sum_{i=1}^{n}(x_i - (\overline X))^2}$$
	
	\item Сравниваются значения
	
	$$|x_{\text{сомнит}} - (\overline X)| \text{ ? } s(x)K_{\text{Ш}}$$
	
	Если $|x_{\text{сомнит}} - (\overline X)|$ больше, то результат отбрасывается, иначе остается.
	
	\item Данные операции следует повторить на полученной выборке до тех пор, пока все значения все не будут удовлетворять критерию Шарлье или в выборке не останется элементов.
\end{itemize}

\section{Описание программного компонента}

В качестве основного языка программирования был выбран Python, графическая часть реализована с помощью HTML + JavaScript + CSS.

\textbf{Python}

Часть на языке Python представляет из себя сервер, который принимает исходную выборку, обрабатывает её и отправляет результат пользователю.

Сервер состоит из следующих функций:

\begin{itemize}
	\item $sharlie(n)$ -- высчитывает коэффициент Шарлье, принмает объем выборки, возвращает коэффициент Шарлье;
	
	\item $statistics(x)$ -- высчитывает среднее и СКО, принимет выборку, возвращает среднее и СКО;
	
	\item $metric(x)$ -- высчитывает среднеквадратичные отклонения, принимет выборку, возвращает массив среднеквадратичных отклонений;
	
	\item $borders(x)$ -- высчитывает границы для подходящих значений, принимет выборку, возвращает верхнюю и нижнюю границы;
	
	\item $find\_bad(x)$ -- сортирует выборку на подходящие значения и выбросы, принимает выборку, возвращает массив с подходящими значениями и массив выбросов;
	
	\item $plt\_to\_base64(x, ok, bad, size=5)$ -- создает график, принимает выброку, подходящие значения, выбросы, размер графика, возвращает график;
	
	\item $calc(x)$ -- основная функция работы алгоритма, принимает первоночальную выборку, возвращает массив для каждого шага, состоящий из хороших значение, выбросов, графика, среднего, СКО и коэффициента Шарлье;
	
\end{itemize}

	Остальная часть отвечает за поднятие сервера, процесса получения запроса и отправки ответа.
	
\textbf{HTML + JavaScript + CSS}

	Данная часть приложения содержит в себе интерфейс, который позволяет загрузить датасет из файла или ввести его вручную. 
	Далее данные передаются на сервер и после получения ответа пошагово отображается работа алгоритма.
	
	Для создания приятного графического интерфейса используется фреймворк Bootstrap.
	
\textbf{Работа алгоритма}
	
	Работа алгоритма начинается с загрузки данных, данные можно загрузить из файла, для этого нужно нажать по кнопке <<Выберете файл>> (рис. \ref{fig:screenshot001}) и выбрать необходимый файл.
	После этого загруженные данные отобразятся в окне ввода данных (рис. \ref{fig:screenshot002}).
	Интерфейс также поддерживает ручной ввод данных, для этого их следует вводить в окно ввода данных (\ref{fig:screenshot002}).
	
	\begin{figure}[H]
		\centering
		\includegraphics[width=0.4\linewidth]{images/screenshot001}
		\caption{Кнопка <<Выберете файл>>}
		\label{fig:screenshot001}
	\end{figure}
	
	
	\begin{figure}[H]
		\centering
		\includegraphics[width=0.7\linewidth]{images/screenshot002}
		\caption{Окно ввода данных}
		\label{fig:screenshot002}
	\end{figure}
	
	Для того, чтобы отправить  данные на сервер и начать работу алгоритма, нужно нажать кнопку <<Вычислить>> (рис. \ref{fig:screenshot003}).
	
	\begin{figure}[H]
		\centering
		\includegraphics[width=0.2\linewidth]{images/screenshot003}
		\caption{Кнопка <<Вычислить>>}
		\label{fig:screenshot003}
	\end{figure}

\section{Результаты обработки данных}

	После обработки данных на экран пошагово выводится подробная информация о результатах промежуточных и окочательных расчетов.
	
	На рисунке \ref{fig:screenshot004} приведены результаты промежуточных расчетов.
	На нем можно выделить график, на котором отображены значения выборки (синим -- допустимые значения, оранжевым -- выбросы).
	Далее выводятся хорошие значения, а после них выбросы.
	В последней строчке находится справочная информация о выборке, Среднее значение, СКО и значение критерия Шарлье. 
	
	\begin{figure}[H]
		\centering
		\includegraphics[width=0.95\linewidth]{images/screenshot004}
		\caption{Промежуточный результат обработки выборки}
		\label{fig:screenshot004}
	\end{figure}

	Подобная информация выводится для каждого шага.	
	Итоговый результат выглядит аналогичным образом (рис. \ref{fig:screenshot005}), разница заключается лишь в том, что на этом шаге нет выбросов.
	
	\begin{figure}[H]
		\centering
		\includegraphics[width=0.85\linewidth]{images/screenshot005}
		\caption{}
		\label{fig:screenshot005}
	\end{figure}
	
	

\section{Библиографический список}

\section{Заключение}

\section{Приложения}


\end{document} % конец документа

